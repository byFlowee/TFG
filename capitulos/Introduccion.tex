%%%%%%%%%%%%%%%%%%%%%%%%%%%%%%%%%%%%%%%%%%%%%%%%%%%%%%%%%%%%%%%%%%%%%%%%
% Plantilla TFG/TFM
% Escuela Politécnica Superior de la Universidad de Alicante
% Realizado por: Jose Manuel Requena Plens
% Contacto: info@jmrplens.com / Telegram:@jmrplens
%%%%%%%%%%%%%%%%%%%%%%%%%%%%%%%%%%%%%%%%%%%%%%%%%%%%%%%%%%%%%%%%%%%%%%%%

\chapter{Introduction}
\label{introduction}


\section{Overview}	
In this bachelor's thesis we will be researching the sim2real field, specifically tackling the object detection problem from a semantic segmentation perspective. In the following chapters we will review and analyze the current state of art of some of the most important datasets, architectures and techniques used for semantic segmentation.

\section{Motivation}

The main motivation behind this project is to further investigate the sim2real and object detection field, specifically the UnrealROX project through semantic segmentation techniques, focusing in developing a user friendly framework for developers to easily generate synthetic data sequences in order to apply deep learning algorithms to real world environments. 

\section{Proposal and Goals}

In this work we propose an extension to the UnrealROX project to automatize the synthetic data generation process as well as a study on how this data can be applied to semantic segmentation architectures to solve real world problems.

