%%%%%%%%%%%%%%%%%%%%%%%%%%%%%%%%%%%%%%%%%%%%%%%%%%%%%%%%%%%%%%%%%%%%%%%%
% Plantilla TFG/TFM
% Escuela Politécnica Superior de la Universidad de Alicante
% Realizado por: Jose Manuel Requena Plens
% Contacto: info@jmrplens.com / Telegram:@jmrplens
%%%%%%%%%%%%%%%%%%%%%%%%%%%%%%%%%%%%%%%%%%%%%%%%%%%%%%%%%%%%%%%%%%%%%%%%

\chapter{Results}
\label{results}
\textit{This chapter goes through the results of our experimentation with the previously described implementations. Section \ref{sec:no_synt} reviews the results when training without synthetic data. Section \ref{sec:with_synth} focuses on training with synthetic data.}

\section{Training with no synthetic data}
\label{sec:no_synt}

\begin{figure}[!ht]
	\centering
	\begin{tikzpicture}
	\begin{axis}[width=10cm,xlabel=Epoch, ylabel=Loss, legend style={at={(0.95,0.95)}}, ymin=0.01, ymax=0.04]
		\addplot table [mark=none, x=epoch, y=train_loss, col sep=comma] {archivos/no-synthetic1.csv};
		\addlegendentry{Training loss};
		
		\addplot table [mark=none, x=epoch, y=val_loss, col sep=comma] {archivos/no-synthetic1.csv};
		\addlegendentry{Validation loss};
	\end{axis}
	\end{tikzpicture}
	\caption{Training and validation loss without synthetic data.}
	\label{fig:loss_no_synthetic}
\end{figure}

\section{Adding synthetic data to the training dataset}
\label{sec:with_synth}

\begin{figure}[!ht]
	\centering
	\begin{tikzpicture}
	\begin{axis}[width=10cm,xlabel=Epoch, ylabel=Loss, legend style={at={(0.95,0.95)}}, ymin=0.01, ymax=0.06]
	\addplot table [mark=none, x=epoch, y=train_loss, col sep=comma] {archivos/mixed.csv};
	\addlegendentry{Training loss};
	
	\addplot table [mark=none, x=epoch, y=val_loss, col sep=comma] {archivos/mixed.csv};
	\addlegendentry{Validation loss};
	\end{axis}
	\end{tikzpicture}
	\caption{Training and validation loss with mixed real and synthetic data.}
	\label{fig:loss_mixed}
\end{figure}
